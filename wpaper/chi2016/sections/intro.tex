\section{Introduction}
Striking the right balance between skimming through newspaper articles, blog posts, tweets, on the one hand, and focusing attention on the most important information on the other, is increasingly tough in a world of limited time \cite{maillart2011} and attention \cite{anham2006economics}. One way to overcome information overflow is by applying filters tailored to individual's past interests \cite{}. However, this algorithmic approach is increasingly criticized for generating positive feedback loops, and so-called filter bubbles, leaving people exposed to an ``echo chamber'' of information \cite{}. Another approach involves optimizing individual exposure to information, such that more documents can be processed in the same amount of time.

Speed reading technologies based on rapid serial visual presentation (RSVP) have been developed to increase the throughput of information delivered to people's eyes \cite{slate2014}. {\bf [say more about current speed-reading technologies]}. People choose the speed in their comfort zone (usually around 125 milliseconds per word) prior to reading the text. If the text appears to be harder than expected a slower speed would have been desirable. On the contrary, an easy or boring text, may not deserve as much time, and speed reading could go significantly faster.

Current speed-reading technologies lack the seamless speed control needed to optimize, in real time, the time spent on each word displayed. Interestingly, the desire to seamlessly control RSVP dates back to the very invention of this technique by early cognitive scientists \cite{}.

Brain computer interfaces (BCI) are well-known for being seamless. \bf {[there is a way to interpret this statement such that it is definitely not true. e.g., even the best BCIs are extremely unreliable. i personally understand what you mean here, but some other phrasing would be better - something that indicates they don't require time to control - they just work, in the moment.]} They have primarily been developed to help heavily motor-impaired recover some communication capabilities, in a way that no motor movement is required from the individual \cite{}. Other types of BCI involve neurofeedback, which is used as a remediation technique for people suffering mainly from the Attention Deficit Hyperactivity Disorder (ADHD) \cite{}. Most consumer-grade applications, fostering ``attention" and ``meditation" training using brainwaves, rely on similar mechanisms \cite{}.

Controlling the speed of RSVP requires that the application process signals almost as fast a it is delivered by EEG signal, which is of the order of hundred milliseconds. Moreover, this processing must be lightweight enough to run on a normal computer or mobile device, without slowing the presentation of words on the reader's screen. To enhance usability, the approach should avoid long and boring calibration procedures, and should be efficient from the start.

We address the problem of online optimization of speed reading, with a lightweight algorithm, which guaranties real-time adaption of the rate of word presentation as a function of cognitive activity as captured by single-channel EEG device.

This paper makes 2 primary research contributions: 

(1) It establishes the {\it feasibility} of speed reading system seamlessly controlled by single-channel EEG signal, and 

(2) it exhibits improvements in {\it understanding, comfort and speed} compared to speed reading with constant rate of word display.


%
%The core problem is the time required to actually read, process and memorize information for future restitution, and RSVP was primarily invented for the purpose of studying the fine-grained memory processes at work when people read text \cite{}, listen to audio streams \cite{}, or are presented with images \cite{}. In the case of language, in particular words displayed one after the other, it appears that the time-gains come from reduced eye movement, as the focus remains in a narrow area where the word is displayed \cite{}. Research in cognitive science shows that words can be presented at as fast as XXX words per second without significant loss of understanding and integration (see Section \ref{} for more detailed review of literature, and precisions regarding the nature of understanding and integration: recall, conceptual understanding, etc.). 
%
%Nevertheless, some texts contain words that are most difficult than others, which require extra memory and conceptual processing after each word (resp. group of words). For instance, a short pause at the end of sentences (resp. paragraphs), considerably helps understanding and recall \cite{}. In other words, the capacity to understand a text stems for an adequate optimization (minimization) of time required for integrating knowledge, which might also differ from one subject to another. This optimization can be made manually (i.e., set the number of words per second) at the level of several texts, of one text, maybe at the level of a paragraph, but hardly at the word level, since the time required to set the pace would eliminate the gains obtained from using RSVP. Also, optimization at the text or paragraph levels requires prior knowledge on the text by the user, which is unpractical since the {\it a priori} goal of RSVP is not consolidating knowledge, but rather going quickly through information.
% 
%We are therefore left with three solutions, which consist in (i) setting an average word pace for all words and all text (this average word pace can be manually set/optimized by the user, (ii) programmatically infer the time required to integrate the meaning of a word \cite{smith2013effect}, or (iii) sharply reduce the cost of optimizing the pace-of-word.
%
%The emergence of consumer-grade Brain Computer Interfaces (cBCI) open new opportunities for such seamless and fine-grained control, beyond medical or lab experimentations. Although usual BCIs rely on medical grade devices, we have previously shown the feasibility of cBCI interface with cheap consumer-grade EEG devices, relying on \textcolor{red}{\bf [complete sentence here]} \cite{}. We shall expand this method, using a continuum of entropy-based attention metrics, to compute in real-time the level of attention around each word presented and control the pace of words accordingly, in a continuous optimization process.
%
%

