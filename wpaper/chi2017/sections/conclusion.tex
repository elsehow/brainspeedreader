\section{Conclusion}
\label{conclusion}
%Recognizing the importance of reading more in less time, while avoiding multi-tasking, a way to do more in less time, yet most often at the cost of less grasp on tasks at hand

In this article, we have proposed, designed and tested a new approach to brain computer interfaces, geared towards handling and controlling a fast-paced flow of information in way that limits diversion from the focal taks and hence, prevents multi-tasking. Our approach is based on a peculiar and seldom phenomenon occurring in the brain -- known as {\it self-regulation}. This phenomenon can be assimilated to independent neurofeedback. We have then tested and benchmarked the so-called {\it brain speed reader} in the context of reading text, and we have found that a large proportion of participants to our experiment successfully achieved self-regulation without prior training for at least one of both treatments proposed. On average they would read twice as fast as with normal reading. We furthermore found that age, text length and speed reading comfort have a negative effect on the achievement of self-regulation. On the contrary, capacity to read fast in a normal setting, and reading pleasure have a positive impact on stability. The pleasure in reading suggests the existence of a reward mechanism when a participant achieves self-regulation. Comprehension is only weakly improved when self-regulation is achieved. 

%``It would elevate the computer to a genuine prosthetic extension of the brain" (Vidal 1973)
%``(to identify appropriate correlates of mental states and decisions in external signals" (Vidal 1973)