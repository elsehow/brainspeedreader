\section{Discussion}
\label{discussion}
Recognizing the importance of reading more in less time, while avoiding multi-tasking, we have put to the test the {\it brain speed reader}, a brain-computer interface (BCI) implementing a rate varying rapid serial visual presentation (RSVP) of text words. We have found that a majority of users who participated in our study could control the brain speed reader. Furthermore, we found that roughly half of participants could self-regulate with two opposite control mechanisms ({\it bsr+} and {\it bsr-}). The achievement of self-regulation is negatively influenced by age, text length, topic familiarity and speed reading comfort. Self-regulation achievement (stability) is however highly positively correlated with reading pleasure. Capacity to provide a meaningful text summary {\it ex-post} (the best way to test for text comprehension) is also positively associated with stability, yet in a weakly significant way.

Our results show that the brain speed reader is an adequate technology to foster fast knowledge integration in a world of abundant information and endless news feeds, while at the same time helping preventing multi-tasking. The design implements self-regulation neurofeedback as a way to ensure that users concentrate on reading the text: If the user does not or cannot concentrate, the RSVP rate will drift towards very slow (resp. fast) word display rates. Multi-tasking is discouraged by design, and reading speed is doubled compared to normal reading.

Self-regulation neurofeedback stems from the capacity by the brain to train and adapt its wave modulations in order to perform specific tasks better \cite{piano_neurofeedback}, or to reach desired mental states \cite{neurofeedback_meditation}.  As a special kind of neurofeedback BCI, the brain speed reader may be help remediate reading disabilities or difficulties associated with a lack of concentration capabilities, which is one consequence of multi-tasking. Because the brain speed reader (BSR) does not require prior calibration, training and use occur concomitantly: Either the user achieves self-regulation quickly and then {\it uses} BSR, or she remains in training mode until she reaches self-regulation. In our experiment, we have set very large boundary for slowest (resp. fastest) RSVP rate to make sure that subjects who reach the boundaries effectively failed achieving self-regulation. In training mode, these RSVP rate boundaries may be tuned and personalized to ensure that the user can reach a compromise between attempts to achieve self-regulation and still reading fast with pleasure while training. While we have not tested medical applications for the brain speed reader, it may help users suffering from dyslexia or attention disorder and hyperactivity disorders (ADHD) improve their reading and their attention.

\subsection{Limitations}
The version of the brain speed reader presented here is a first attempt to harness self-regulated neurofeedback for the sake of upgrading the reading experience to arising challenges in society, such as tackling the abundance of natural language- written information while reducing multi-tasking. Although our first experiment show encouraging results, we have come up with the simplest possible implementation and experimental protocol. A number of outstanding questions remain and deserve further work, starting with the hardware we used. Neurofeedback most often involves a limited number of EEG electrodes, but typically of higher quality and on other positions in the 10-20 system. One may want to experiment with a variety of electrode quality, number, and positions to elicit a better hardware configuration.

Our study bears a number of limitations regarding comprehension and memory. We did not find a significant difference of comprehension between RSVP speed read at constant rate versus {\it bsr+} and {\it bsr-} treatments. We find a slight comprehension improvement when users achieve more stable self-regulation, but this result need further validation, presumably with a larger data set. We have also not measured how comprehension and word recall are diminished (resp. increased) as reading speed doubles, i.e., when comparing brain speed reading and reading text normally. Previous research suggests that comprehension is diminished by constant RSVP rate speed reading \cite{kujala2007phase}, although it remains the same or is slightly improved for people with reading disabilities. Further investigation is required to better elicit how self-regulation stability influences comprehension and word recall

We shall also investigate why some participants could not achieve self-regulation: It remains unclear if they had indeed to train further before making it right, or if the BSR parameters we chose [initial RSVP rate $X(t=0) = 125$ ms/word and $|\alpha| = 0.005$] may only be suitable for a subset of the population, and may require further adaptation. Similarly, we have no information on the time it takes to learn given that the user could not achieve self-regulation at once. We chose $X(t=0) = 125$ ms/word as the starting RSVP rate, which is also the value known to be most comfortable on average to people who read with RSVP. To further understand and validate the self-regulation mechanism, it would be desirable to perform additional investigations with a variety of starting rates. Our hypothesis is that within a yet unknown RSVP rate range, users capable of self-regulation can stabilize the rate, while beyond some limits it is impossible to control the brain speed reader, in particular when the presentation speed is high. It also remains to be seen if users tend to stabilize the rate close to 125ms/word on average, even if the initial RSVP rate is smaller (resp. larger).

Even though there is no need for preliminary {\it ad-hoc} calibration to use the brain speed reader, learning still occurs ``on the fly", with several advantages as already mentioned above. It would nevertheless be desirable to better understand how the user transitions from {\it learning} to {\it using}. This has implications for more elaborate and personalized UX design.

Finally, we found that text length has a negative effect on self-regulation stability (roughly one unit per 1300 words as shown in Table \ref{tab:reg}). This is a concern because the brain speed reader is supposed to let users avoid multi-tasking to concentrate on a single task instead. Here, it appears that if the task lasts too long, then self-regulation undergo changes of regime and reduced stability. This suggests that either BSR is suited for {\it long but not that long tasks} and/or there is another factor involving learning how to keep self-regulation highly stable over long tasks. This is has further implications on comprehension and memory, which need to be further addressed.

\subsection{Beyond speed reading}
Here, we have designed and tested brain-computer interface (BCI), which helps seamlessly control the parsing rate of a coherent sequence of a words as visual stimuli. The same time varying rate RSVP coupled with a neurofeedback BCI may be used beyond reading. It could apply similarly to cartoons, video or audio streams, if the control is made on the continuous speed (i.e., the pitch) of the media stream, similarly to the (discrete) presentation rate used in our experiment.

Similar BCI could be used to assess the quality and the coherence of a piece of text. Indeed, it appears that when reading, the brain makes some heuristic predictions of what word(s) is (resp. are) coming next \cite{smith2013effect}. Unexpected words trigger an unusual cognitive activity which slows the reading. Recording how the effects of unexpected stimuli materialize in conjunction with control remain yet to the be investigated. 
