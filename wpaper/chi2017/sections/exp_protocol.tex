\section{Experimental protocol \& participant demographics}
\label{exp_protocol}
We put the brain speed reader to test with a standardized experiment involving 21 participants. The IRB approved experimental protocol is the following.
%\footnote{UC Berkeley Committee for the Protection of Human Subjects (CPHS Number : 2014-04-6247 approved on June 20, 2014).} 
Upon arrival each participant is briefed on the purpose of the experiment, and gets set up by wearing a Neurosky Mindwave EEG headset. Signal quality checks are automatically performed before the experiment and between each task. If signal quality is not sufficient, the experimenters help the participant adjusting the headset. The experiment starts with 5 preliminary tasks: (i) blink five times, (ii) stay quiet with eyes open for 15 seconds (nothing is displayed on the screen), (iii) stay quiet with eyes closed for 15 seconds, (iv) perform 5 multiplications and (v) read a text entirely displayed on the screen. Then, 4 {\it speed reading} RSVP treatments are randomly assigned: constant rate (repeated), brain speed reader plus (bsr+, i.e., $\alpha = 0.005$) and brain speed reader minus (bsr-, i.e., $\alpha = -0.005$). For each treatment, one text is chosen from 6 news articles of various lengths and topics (e.g., politics, environmental issues, justice, technology, energy savings, food). To minimize latencies, words are presented directly on the bash terminal window of a powerful Apple computer, equipped with Mac OS, and the portion of the code in charge of displaying words is run in a thread isolated from the signal processing and rate calculation thread. After each treatment, participants take 3 comprehension questions: (a) write a short summary, (b) free recall of proper nouns, (c) pick from a list common nouns seen in the text. The summary (question a) is graded on a scale from 0 to 10 by three reviewers. The summary grade is the average of the three independent assessments. Recall questions (a and b) are graded automatically. At the end of the experiment final demographic questions are asked.

{\bf [add some demographics]}