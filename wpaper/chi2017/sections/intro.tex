\section{Introduction}
\label{intro}
With real-time information, grasping, filtering and retaining knowledge has become a critical issue for most people connected to the Internet. We are constantly bombarded with competing feeds of Tweets, Facebook posts, WhatsApp and Snapchat messages, emails, or news articles. This abundant and endless flow of information has become a challenge for most connected human beings. In this overwhelming environment, most people struggle managing their time \cite{maillart2011} and processing information fast enough. 

When subjected to competing attention \cite{hansen2001competing} from a variety of newsfeeds and other online platforms \cite{foehr2006media,harris2016}, it has been found that people engage into multi-tasking \cite{wallis2006multitasking}. {\it Prima facie} multi-tasking helps save time: A 2005 survey by the Kaiser Family Foundation found that Americans aged from 8 to 18 could pack 8.5 hours' worth of media in 6.5 hours a day, by exposing themselves to multiple media simultaneously \cite{roberts2005generation}. However, there is evidence that multi-tasking has significant implications for the way people learn, reason, socialize: The quality of output and depth of thought deteriorates as one attends to ever more tasks \cite{hembrooke2003laptop}, and, at best, there is a trade-off of what can be learned from different competing tasks performed at the same time. The negative effects of multi-tasking are more important on tasks which require high engagement and concentration \cite{wallis2006multitasking}. These results are further confirmed by neuroscience \cite{burgess2000strategy}.

Here, we ask if information overflow, competing attention, as well as resulting multi-tasking and loss of concentration skills may be alleviated all at once with the help of a new kind of brain-computer interface (BCI), geared toward handling and controlling the pace of a large volume of information flow. Our approach is reminiscent of various techniques used to go quickly through information.

Back in 1950's, Evelyn Wood proposed a {\it reading dynamics} method to read quickly  \cite{frank1994evelyn}. This method  based on tracing lines of text became immensely popular and U.S. Presidents Kennedy and Carter learned and used this technique. Other techniques include {\it skimming} (i.e., searching the sentences of a page for clues to meaning), {\it scanning} (i.e.,  actively looking for information using mind-maps) and {\it meta-guiding} (i.e., drawing invisible shapes on a page of text in order to broaden the visual span for speed reading). However, these techniques often lead to comprehension rates way below 50\% because they involve reading only a portion of the text \cite{carver1992reading}. They are indeed rather used to go quickly through a text, before deciding to invest more time to read it in details. On the contrary to partial reading, some speed reading techniques involve exhaustive reading: They mostly borrow from rapid serial visual presentation (RSVP) \cite{potter1984rapid}, a technique used early on to investigate cognition mechanisms, such as memory processes \cite{potter1999understanding}, and the attention blink \cite{shapiro1994attention}. RSVP speed reading consists in presenting words of a text one at a time at high pace (usually less than 150 milliseconds per word) and in the middle of a empty screen. The reading speed gain is essentially mechanical: The eyes remain always focused on the same spot and thus don't have to continuously move and re-focus \cite{slate2014}.

Although all speed reading these techniques ensure fast information processing, either by random sampling or by fast-pace sequential exposure, none of them provide a mechanism to achieve fast and exhaustive -- yet comfortable -- reading, and at the same time, ensure engagement, or at least, a training mechanism to help achieving engagement. Here, we present a brain-computer interface (BCI) that addresses the challenge of going through information quickly while helping users maintain a high attention level. This brain-computer interface, also called {\it brain speed reader (BSR)}  allows going through a coherent sequence of information (e.g., words in a news article) at fast pace using a brain controlled rapid serial visual presentation (RSVP). This interface furthermore leverages the peculiar ability by the brain to {\it self-regulate} in presence of neurofeedback \cite{lubar1976eeg,weiskopf2004self}.

This paper is organized as follows. We first review the relevant background literature, namely on brain computer interfaces and self-regulated neurofeedback. We then present the {\it brain speed reader} apparatus and its key design choices. as well as the standardized experiments we conducted. We then show how users achieve a good control of the brain speed reader through balanced neurofeedback regulation, how the demographics and the context of the text influence self-regulation and in turn, how self-regulations influences comprehension. We finally discuss our results, identify limitations and conclude.





%From an information processing viewpoint, one main factor hindering fast reading is sub-vocalization (mental or auditory reading), which should be replaced by visual reading (understanding the meaning of the word, rather than sounding or hearing). Achieving visual reading is not easy and requires manual training \cite{visual_read_manual}, or with the help of software \cite{visual_read_software}. 



%Competing attention is associated with the fact that although the human brain needs some time to immerse itself in a specific environment in order to tackle a task involving cognition (strategy application disorder) \cite{} , . 


%At the notable exception of so-called  ``supertaskers" ($\approx 2.5\%$ of a population) \cite{watson2010supertaskers}, individuals handle multitasking badly while the world of information heavily encourages multitasking through computers, smartphones, and the myriad of news, chat, news, and social media applications and feeds \cite{}. Heavy media multitaskers perform worse that low media multitaskers  on the primary task. The former rely more on a bottom up approach to information procesing while the latter group has a more top-down (controlled) approach to information processing \cite{ophir2009cognitive}.

%It is still difficult to assert that multitasking is worse that mono-tasking, the former having perhaps some still unknown advantages, which may help strive in a highly soliciting media world. But offering the possibility to quickly switch between heavy multitasking to monotasking may be combine the best of both worlds, and quickly adapt to fast changing challenges. We propose that technology inspired from neuro-feedback could help individuals or enjoy mono-tasking -- or remediate from multi-tasking -- in a seamless way.



%Striking the right balance between skimming through newspaper articles, blog posts, tweets, on the one hand, and focusing attention on the most important information contents on the other hand, is an increasingly though challenge in a world of limited time \cite{maillart2011} and attention \cite{anham2006economics}. One way can overcome information overflow by applying filters tailored on individual's past interests \cite{}. This algorithmic approach is however increasingly criticized for generating positive feedback loops and so-called filter bubbles, leaving people exposed to more of the same information \cite{}. Another approach consists in optimizing individual exposure to information, in a way that more knowledge can be processed for the same amount of time.
%
%Speed reading technologies based on rapid serial visual presentation (RSVP) have been developed to increase the throughput of information delivered to people's eyes \cite{slate2014}. {\bf [say more about current speed-reading technologies]}. People choose the speed in their comfort zone (usually around 125 milliseconds per word) prior to reading the text. If the text appears to be harder than expected a slower speed would have been desirable. On the contrary, an easy or boring text, may not deserve as much time, and speed reading could go significantly faster.
%
%Current speed-reading technologies lack the seamless speed control needed to provide online optimization of time spent on each word displayed, or at least on portions of sentences and paragraphs. Interestingly, the wish to seamlessly control RSVP dates back to the very invention of this technique by early cognitive scientists \cite{}.
%
%Brain computer interfaces (BCI) are well-known for being seamless. They have primarily been developed to help heavily motor-impaired recover some communication capabilities, in a way that no physical input is required from the individual \cite{}. Other types of BCI involve neurofeedback, which is used as a remediation technique for people suffering mainly from the Attention Deficit Hyperactivity Disorder (ADHD) \cite{}. Most consumer-grade applications, fostering ``attention" and ``meditation" training using brainwaves rely on similar mechanisms \cite{}.
%
%Controlling speed reading requires however to process signal almost as fast a it is delivered by EEG signal, which is of the order of hundred milliseconds. Moreover, this processing must be lightweight enough to run on a normal computer or mobile device, without slowing the presentation of words on the reader's screen. To enhance usability, the approach should avoid long and boring calibration procedures, and should be efficient from the start.
%
%We address the problem of online optimization of speed reading, with a lightweight algorithm, which guaranties real-time adaption of the rate of word presentation as a function of cognitive activity as captured by single-channel EEG device.
%
%This paper makes 2 primary research contributions: 
%
%(1) It establishes the {\it feasibility} of speed reading system seamlessly controlled by single-channel EEG signal, and 
%
%(2) it exhibits improvements in {\it understanding, comfort and speed} compared to speed reading with constant rate of word display.
%

%
%The core problem is the time required to actually read, process and memorize information for future restitution, and RSVP was primarily invented for the purpose of studying the fine-grained memory processes at work when people read text \cite{}, listen to audio streams \cite{}, or are presented with images \cite{}. In the case of language, in particular words displayed one after the other, it appears that the time-gains come from reduced eye movement, as the focus remains in a narrow area where the word is displayed \cite{}. Research in cognitive science shows that words can be presented at as fast as XXX words per second without significant loss of understanding and integration (see Section \ref{} for more detailed review of literature, and precisions regarding the nature of understanding and integration: recall, conceptual understanding, etc.). 
%
%Nevertheless, some texts contain words that are most difficult than others, which require extra memory and conceptual processing after each word (resp. group of words). For instance, a short pause at the end of sentences (resp. paragraphs), considerably helps understanding and recall \cite{}. In other words, the capacity to understand a text stems for an adequate optimization (minimization) of time required for integrating knowledge, which might also differ from one subject to another. This optimization can be made manually (i.e., set the number of words per second) at the level of several texts, of one text, maybe at the level of a paragraph, but hardly at the word level, since the time required to set the pace would eliminate the gains obtained from using RSVP. Also, optimization at the text or paragraph levels requires prior knowledge on the text by the user, which is unpractical since the {\it a priori} goal of RSVP is not consolidating knowledge, but rather going quickly through information.
% 
%We are therefore left with three solutions, which consist in (i) setting an average word pace for all words and all text (this average word pace can be manually set/optimized by the user, (ii) programmatically infer the time required to integrate the meaning of a word \cite{smith2013effect}, or (iii) sharply reduce the cost of optimizing the pace-of-word.
%
%The emergence of consumer-grade Brain Computer Interfaces (cBCI) open new opportunities for such seamless and fine-grained control, beyond medical or lab experimentations. Although usual BCIs rely on medical grade devices, we have previously shown the feasibility of cBCI interface with cheap consumer-grade EEG devices, relying on \textcolor{red}{\bf [complete sentence here]} \cite{}. We shall expand this method, using a continuum of entropy-based attention metrics, to compute in real-time the level of attention around each word presented and control the pace of words accordingly, in a continuous optimization process.
%
%

