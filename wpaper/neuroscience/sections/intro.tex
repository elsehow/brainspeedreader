\section{Introduction}
Neurofeedback has become an increasingly popular approach \cite{evans1999introduction}, used for the remediation of a variety of brain-related disorders, including attention deficit hyperactivity disorder 
\cite{lubar1995evaluation,gevensleben2009neurofeedback,arns2009efficacy}, depression \cite{saxby1995alpha}, epilepsy \cite{sterman2006foundation}, and autism \cite{jarusiewicz2002efficacy,kouijzer2009neurofeedback,coben2010neurofeedback}, as well as for the condition improvement of veterans with post-traumatic syndrom disorder \cite{peniston1991alpha}. Neurofeedback is also used for the enhancement of cognitive performances \cite{vernon2003effect,hanslmayr2005increasing,gruzelier2014eeg}, from children \cite{} to elderly \cite{angelakis2007eeg}, and for task specific purposes, such as musical performance \cite{egner2003ecological}, sport \cite{wilson2006mind}, for emotions \cite{raymond2005effects,gruzelier2014eeg}, for mood \cite{raymond2005effects}, or for meditation \cite{gruzelier2009theory,rubik2011neurofeedback,brandmeyer2013meditation}.\\

Doubts about neurofeedback \cite{beyerstein1990brainscams,vernon2009alpha}


The main approach to neurofeedback consists of identifying a frequency range of the EEG signal -- usually {\it alpha} ( X-X Hz), {\it beta} (X-X Hz) or {\it theta} (4 - 7 Hz) and sensorimotor rhythm  (SMR) (low beta 12-15 Hz), identify tasks that are known to have an effect on the activation of these frequency ranges, and then {\it train} people with these tasks until they reach a frequency activation target, known to be beneficial for the desired remediation or cognitive enhancement.\\

{\bf testing neurofeedback:} EEG signature and phenomenology of alpha/theta neurofeedback training versus mock feedback \cite{egner2002eeg}

Other approaches to neurofeedback include online learning ...\\

And neurofeedback is also used to locate areas of interest in the brain (usually areas of interest in relation to the disorder and the cognitive performance). This type of neurofeedback is usually performed with fMRI.\\












\clearpage

Reading out, or decoding, mental content from brain activity is a challenging problem in neuroscience. Recent functional magnetic resonance imaging (fMRI) studies have suggested the possibility to reconstruct visual experience from brain measurements\cite{kay2008identifying,naselaris2009bayesian,nishimoto2011reconstructing}. This research typically involves in depth studies of the brain circuitry with cutting-edge equipment, which comes at the expense of scalable and reproducible identification of human activity in naturalistic environments. Instead, we focus on decoding human activity from low quality signal delivered by consumer-grade EEG headsets in a rapid serial visual presentation (RSVP) setting. The development of such lightweight and affordable technique is critical for consumer ready applications, such as special brain-computer interfaces to handle large flows of information.



``Similarly there are reviews on the burgeoning approachbranded Brain Computer Interface (BCI), one typically focussing onthe neuro-rehabilitation of patients with tetraplegia from brain orspinal cord injuries enabling patients through EEG feedback to com-municate and/or physically interact with their environment, and upuntil now largely consisting of engineering innovations and a smallclinical evidence base (Birbaumer et al., 2008; Mak and Wolpaw,2009; Silvoni et al., 2011)."
 
``As will be seen a diversity of neurofeedback training protocolshas been applied for optimising performance. The most popu-lar one has involved training-up the amplitude of the SensoryMotor Rhythm (SMR) 12�15 Hz band while inhibiting outer-lyingbands in the EEG spectrum." 

``By extrapolating toADHD the potential of reducing the excitability of the sensorimotor system with concomitant suppression of theta, and followingSMR training with training up adjacent low beta activity (16�22 Hz;beta1) an index of EEG desynchronisation, improvements in attention and hyperactivity were first demonstrated in case studies andearly controlled trials (e.g., Lubar and Shouse, 1976; Lubar et al.,1995a; Rossiter and LaVaque, 1995; Linden et al., 1996) and nowhave a substantive evidence base (Monastra et al., 2005; Arns et al.,2009; Lofthouse et al., 2012)." $\rightarrow$  suppression of SMR variations.

``Another pioneering protocol involved raising the theta-alpharatio with auditory feedback and eyes closed, termed alpha/theta(A/T) training." $\rightarrow$  A/T training

``Historically self-regulation of slow cortical poten-tials (SCPs) had received extensive validation (Elbert et al., 1984;Rockstroh, 1989), but has not attracted interest in the optimalperformance field despite promise in trials with ADHD where it has been compared favourably with EEG-spectrum training(Gevensleben et al., 2009)."

neurofeedback typically involves training for increasing a frequency band on a given electrode in the 10-20 system, and then assess performance improvements on a variety of tasks: For instance ....

With two outcomes: refined strategies for neurofeedback training AND better understanding of frequency band activations on performance.

However, each brain is unique somehow, with different cognitive strategies to handle problems. It is unclear if a cognitive function can be narrowed to a frequency band and spatial position for all people. For instance, some research has found that power spectrum frequency bands may vary from one subject to another \cite{}.

One way to think out of the box, consists in taking one electrode position (here, Fp9), assuming that the EEG signal exhibits high correlation across positions, at least neighboring positions, and making no assumption on the frequency of interest, and to let the ``data speak". For that, a high throughput is needed (define better what is meant by throughput), with a system that scales. 

deductive vs. inductive approach. 

Our setup allows testing if there is a generalizable adaptation mechanism common to a majority of brains /people, or if on the contrary, if each brain / cog process is somehow unique. These are crucial questions to address in order to understand the mechanisms of neuro-feedback





Here, we intentionally employ a simple, even {\it rudimentary} setting: one EEG electrode on Fp9, with a mass market EEG headset (neurosky mind wave). We then design a stringent and fast paced neurofeedback mechanism, including rapid serial visual presentation (RSVP) of a text (word by word at display rate approx. 150 milliseconds/word). We invite participants for a short, yet intense session of approx. 15 mins, in which they are presented 3-4 texts of size ranging from XXX to XXX words, we various treatments.

Here, we do not focus {\it per se} on performance, but on the ability to cope, with the intense RSVP set of stimuli
. Here, ``training" occurs within seconds: Either the participant manages to control the word pace, or he/she cannot.




Striking the right balance between skimming through newspaper articles, blog posts, tweets, on the one hand, and focusing attention on the most important information contents on the other hand, is an increasingly though challenge in a world of limited time \cite{maillart2011} and attention \cite{anham2006economics}. One way can overcome information overflow by applying filters tailored on individual's past interests \cite{}. This algorithmic approach is however increasingly criticized for generating positive feedback loops and so-called filter bubbles, leaving people exposed to more of the same information \cite{}. Another approach consists in optimizing individual exposure to information, in a way that more knowledge can be processed for the same amount of time.

Speed reading technologies based on rapid serial visual presentation (RSVP) have been developed to increase the throughput of information delivered to people's eyes \cite{slate2014}. {\bf [say more about current speed-reading technologies]}. People choose the speed in their comfort zone (usually around 125 milliseconds per word) prior to reading the text. If the text appears to be harder than expected a slower speed would have been desirable. On the contrary, an easy or boring text, may not deserve as much time, and speed reading could go significantly faster.

Current speed-reading technologies lack the seamless speed control needed to provide online optimization of time spent on each word displayed, or at least on portions of sentences and paragraphs. Interestingly, the wish to seamlessly control RSVP dates back to the very invention of this technique by early cognitive scientists \cite{}.

Brain computer interfaces (BCI) are well-known for being seamless. They have primarily been developed to help heavily motor-impaired recover some communication capabilities, in a way that no physical input is required from the individual \cite{}. Other types of BCI involve neurofeedback, which is used as a remediation technique for people suffering mainly from the Attention Deficit Hyperactivity Disorder (ADHD) \cite{}. Most consumer-grade applications, fostering ``attention" and ``meditation" training using brainwaves rely on similar mechanisms \cite{}.

Controlling speed reading requires however to process signal almost as fast a it is delivered by EEG signal, which is of the order of hundred milliseconds. Moreover, this processing must be lightweight enough to run on a normal computer or mobile device, without slowing the presentation of words on the reader's screen. To enhance usability, the approach should avoid long and boring calibration procedures, and should be efficient from the start.

We address the problem of online optimization of speed reading, with a lightweight algorithm, which guaranties real-time adaption of the rate of word presentation as a function of cognitive activity as captured by single-channel EEG device.

This paper makes 2 primary research contributions: 

(1) It establishes the {\it feasibility} of speed reading system seamlessly controlled by single-channel EEG signal, and 

(2) it exhibits improvements in {\it understanding, comfort and speed} compared to speed reading with constant rate of word display.


%
%The core problem is the time required to actually read, process and memorize information for future restitution, and RSVP was primarily invented for the purpose of studying the fine-grained memory processes at work when people read text \cite{}, listen to audio streams \cite{}, or are presented with images \cite{}. In the case of language, in particular words displayed one after the other, it appears that the time-gains come from reduced eye movement, as the focus remains in a narrow area where the word is displayed \cite{}. Research in cognitive science shows that words can be presented at as fast as XXX words per second without significant loss of understanding and integration (see Section \ref{} for more detailed review of literature, and precisions regarding the nature of understanding and integration: recall, conceptual understanding, etc.). 
%
%Nevertheless, some texts contain words that are most difficult than others, which require extra memory and conceptual processing after each word (resp. group of words). For instance, a short pause at the end of sentences (resp. paragraphs), considerably helps understanding and recall \cite{}. In other words, the capacity to understand a text stems for an adequate optimization (minimization) of time required for integrating knowledge, which might also differ from one subject to another. This optimization can be made manually (i.e., set the number of words per second) at the level of several texts, of one text, maybe at the level of a paragraph, but hardly at the word level, since the time required to set the pace would eliminate the gains obtained from using RSVP. Also, optimization at the text or paragraph levels requires prior knowledge on the text by the user, which is unpractical since the {\it a priori} goal of RSVP is not consolidating knowledge, but rather going quickly through information.
% 
%We are therefore left with three solutions, which consist in (i) setting an average word pace for all words and all text (this average word pace can be manually set/optimized by the user, (ii) programmatically infer the time required to integrate the meaning of a word \cite{smith2013effect}, or (iii) sharply reduce the cost of optimizing the pace-of-word.
%
%The emergence of consumer-grade Brain Computer Interfaces (cBCI) open new opportunities for such seamless and fine-grained control, beyond medical or lab experimentations. Although usual BCIs rely on medical grade devices, we have previously shown the feasibility of cBCI interface with cheap consumer-grade EEG devices, relying on \textcolor{red}{\bf [complete sentence here]} \cite{}. We shall expand this method, using a continuum of entropy-based attention metrics, to compute in real-time the level of attention around each word presented and control the pace of words accordingly, in a continuous optimization process.
%
%

