\section{Introduction}

%Striking the right balance between skimming through newspaper articles, blog posts, tweets, on the one hand, and focusing attention on the most important information contents on the other hand, is an increasingly though challenge in a world of limited time \cite{maillart2011} and attention \cite{anham2006economics}. One way can overcome information overflow by applying filters tailored on individual's past interests \cite{}. This algorithmic approach is however increasingly criticized for generating positive feedback loops and so-called filter bubbles, leaving people exposed to more of the same information \cite{}. Another approach consists in optimizing individual exposure to information, in a way that more knowledge can be processed for the same amount of time.




From early research on operational conditioning of cats \cite{wyrwicka1968instrumental,sterman1969electrophysiological}, neurofeedback has been found to ``train the brain" in order to remediate brain disorders, such as for attention deficit hyperactivity disorder \cite{lubar1976eeg,monastra2006electroencephalographic},  depression \cite{saxby1995alpha},  post-traumatic syndrom disorder \cite{peniston1991alpha}, or autism \cite{kouijzer2009neurofeedback,coben2010neurofeedback}, to reduce the incidence of epileptic seizures \cite{sterman2006foundation}. Neurofeedback is also thought to enhance cognitive performance, namely for music \cite{egner2003ecological}, sport \cite{wilson2006mind}, control of emotions \cite{gruzelier2014eeg} and mood \cite{raymond2005effects}, or to help the practice of meditation \cite{gruzelier2009theory,rubik2011neurofeedback,brandmeyer2013meditation}. Because the mechanisms of neurofeedback remain unclear, the validity of theories and results remain questioned\cite{beyerstein1990brainscams,vernon2009alpha}.\\

Most studies consider brain activity as a variable dependent of exogenous stimulations  (e.g., visual or auditory stimuli) and measure brain response and adaptation to these well calibrated stimuli (resp. behaviors), with focus on electroencephalogram (EEG) frequency bands, such as increase the sensorimotor rhythm (SMR, 12-14 Hz band) \cite{}, alpha band (7.5 - 12.5 Hz) \cite{}, beta band (15 - 20 Hz) \cite{}, theta band (6-10 Hz), gamma band (25 -100 Hz) \cite{}, and a multitude increased / decreased band activity combinations. However, these frequency bands and their activations are individual specific, and are dependent to a number of factors, including age \cite{} and cognition training following exposure to stimuli \cite{}. As a method intended to train specific frequency bands (i.e., change their intensity following stimuli or behaviors), neurofeedback embodies this plasticity, but it generally assumes that frequency bands are invariant across subjects. Therefore, most experiments target and train pre-determined frequency bands. In contrast, neurofeedback experiments can also use self-regulated brain activity to study voluntary controlled behaviors, and conversely, the nature of brain activity can be unveiled from observed behaviors. Functional magnetic resonance imaging (fMRI) is a powerful tool for source localization, and methods have been developed recently for real-time fMRI (rtfMRI), which allow exposing subjects to brain triggered stimuli (audio or video) and at the same time map brain activity \cite{}. Feedback obtained from rtfMRI is slow however (between 1 and 2 seconds) as fMRI measures the blood oxygenation level dependent signal \cite{ogawa1990brain,ogawa1990oxygenation}. Feedback speed may also be an issue with EEG, depending on the amount of required signal processing (e.g., artifact removal, blind source separation), which is highly dependent on the quality of the equipment used and the desired resolution, in particular for source localization.\\

%\paragraph{\bf fMRI and neurofeedback} real-time fMRI \cite{weiskopf2004self,weiskopf2012real,sulzer2013real} $\rightarrow$ time resolution is slow $>1$ second (delay $<2$ s) .
{\bf add a paragraph on BCI, as a system with computer as the dependent variable, on the contrary neurofeedback where the dependent variable is the brain}


Brain activity is commonly self-regulated in number of tasks performed in everyday's life: For instance, people modulate their reading patterns as a function of cognitive activity associated with retrieving the meaning of words from long-term memory \cite{federmeier1999rose} and keeping a general sense of a sentence in short-term working memory \cite{jackson1979processing}. Sensory and cognitive determinants of reading speed \cite{jackson1975sensory}. The reading speed is for instance affected among other things by the length of words, because longer words have more characters on the one hand (more bits need to be processed), and because they are less frequent in the corpus \cite{jackson1979processing}. Word length and the structure of short-term memory \cite{baddeley1975word}  {\bf [nb: reading speed is a big topic is cognitive science, maybe it deserves a larger paragraph]}.\\

Like number of other tasks, reading requires handling a flow of information (a coherent sequence of visual stimuli), and making sense of it. The latter processing requires variable time, and thus, the reading pace is self-regulated to account for the local difficulty of word sequences, to varying interest in the text read, and under time constraints for productivity or competing attention reasons.\\

%Current speed-reading technologies lack the seamless speed control needed to provide online optimization of time spent on each word displayed, or at least on portions of sentences and paragraphs. Interestingly, the wish to seamlessly control RSVP dates back to the very invention of this technique by early cognitive scientists \cite{}.

Despite the ubiquitous importance of cognitive functions and their effects on people's schedule, how people self-regulate their brain activity for the achievement of tasks, remains unclear. Using a standardized experiment combining the presentation of newspaper articles word by word with rapid serial visual presentation (RSVP), which rate of presentation is controlled by a simple neurofeedback apparatus (see Figure \ref{} and Method section), we find that the majority of participants achieve a high level of self-regulation. However, our results show that there is no unique pattern, and each participant develops her own self-regulation strategy, which falls into four broad strategy categories. For each strategy, we inspect which frequency bands get more activated as a function of word sequence difficulty. Conversely, the inability to self-regulate the pace of words with the brain stems from {\bf the lack of stable strategy}. \\



