\section{Method}

Twenty-one healthy subjects aged 18 to 55 participated in this study. Each participant read 4 texts, randomly selected out of 6 newspaper articles (389-990 words per text). The texts were displayed through rapid serial visual presentation (RSVP) \cite{potter1984rapid,potter1975time}, while measuring brain activity with the cheapest consumer grade EEG headset available on the market (Neurosky Mindwave,  $\approx$ \$100). Neurosky device collects EEG signal on the left forehead (Fp9 position in the 10-20 system) with a dry electrode. For each new word displayed on the screen, the Shannon entropy of the power spectrum of the signal \cite{Tellenbach2009Beyond} was computed out of the last 512 voltage measures (i.e., one second of EEG data at a 512 Hz sampling rate). The Shannon entropy is used here as a powerful method to compress the information contained in the power spectrum (i.e., a vector with 512 values) into a scalar value \cite{ornstein1993entropy}.  \\

Three randomized treatments were applied:  (i) constant RSVP rate ({\it rate} = 125 milliseconds per word, applied to 2 over 4 texts), (ii) RSVP {\it rate}  increases with higher entropy and conversely, (iii) RSVP {\it rate}  decreases with higher entropy (see Figure 1 for experiment setting). For treatments (ii) and (iii), if the participant cannot control RSVP with her brain activity, the {\it rate} drifts away, very slow or very fast.\\

Here, we consider the EEG signal when the participant is subjected to a coherent source of sequential stimuli (i.e., words of a text presented one at the time in a sequential order). The brain decoding procedure aims at identifying the text from the sequence of entropy measures computed from the participant's EEG signal.\\
