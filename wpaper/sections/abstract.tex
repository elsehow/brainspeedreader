\begin{abstract}
Back in the 70's \cite{forster1970visual} and the 80's \cite{}, Rapid Serial Visual Presentation (RSVP) attracted the attention of linguists and cognitive scientists, in order to understand how language is processed, memorized and stored by the brain. Researchers found that text presented with RSVP could be as well understood as when read in a standard way, although the reading pace could be faster \textcolor{red}{\bf [substantiate this sentence with figures]} \cite{}. Researchers noted that RSVP could become a mainstream reading technique if the pace of word presentation could be controlled in a seamless way (i.e., at no time cost) by the reader \cite{potter1984rapid}. Modern consumer-grade Brain Computer Interfaces (cBCI) \cite{} offer this seamless control on the pace of RSVP. Using a cohort of XX subjects, we test an apparatus called {\it brain speed reader}, which combines RSVP and control through cBCI. We show that the reading performances \textcolor{red}{\bf [define performance]} and comfort are significantly increased \textcolor{red}{\bf [to be verified]} compared to a constant word-pace Rapid Serial Visual Presentation. 
\end{abstract}