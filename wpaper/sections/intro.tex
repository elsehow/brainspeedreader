\section{Introduction}
Dating back to the 80s', Rapid Serial Visual Presentation (RSVP) has recently received a hype of attention through social media \cite{slate2014}. With the explosion of content on the Web, people are increasingly under pressure to incorporate more information, which they shall decide to either internalize the knowledge for future reuse, or on the contrary, dismiss useless information. Like many other tasks nowadays, reading faces a problem of economy of time as non-storable scarce resource \cite{maillart2011}, and given the importance of integrating knowledge, optimizing reading time could help people either (i) save tens of minutes, if not hours, every day, or conversely, (ii) allow read more for the same time budget.

The core problem is the time required to actually read, process and memorize information for future restitution, and RSVP was primarily invented for the purpose of studying the fine-grained memory processes at work when people read text \cite{}, listen to audio streams \cite{}, or are presented with images \cite{}. In the case of language, in particular words displayed one after the other, it appears that the time-gains come from reduced eye movement, as the focus remains in a narrow area where the word is displayed \cite{}. Research in cognitive science shows that words can be presented at as fast as XXX words per second without significant loss of understanding and integration (see Section \ref{} for more detailed review of literature, and precisions regarding the nature of understanding and integration: recall, conceptual understanding, etc.). 

Nevertheless, some texts contain words that are most difficult than others, which require extra memory and conceptual processing after each word (resp. group of words). For instance, a short pause at the end of sentences (resp. paragraphs), considerably helps understanding and recall \cite{}. In other words, the capacity to understand a text stems for an adequate optimization (minimization) of time required for integrating knowledge, which might also differ from one subject to another. This optimization can be made manually (i.e., set the number of words per second) at the level of several texts, of one text, maybe at the level of a paragraph, but hardly at the word level, since the time required to set the pace would eliminate the gains obtained from using RSVP. Also, optimization at the text or paragraph levels requires prior knowledge on the text by the user, which is unpractical since the {\it a priori} goal of RSVP is not consolidating knowledge, but rather going quickly through information.
 
We are therefore left with three solutions, which consist in (i) setting an average word pace for all words and all text (this average word pace can be manually set/optimized by the user, (ii) programmatically infer the time required to integrate the meaning of a word \cite{smith2013effect}, or (iii) sharply reduce the cost of optimizing the pace-of-word.

The emergence of consumer-grade Brain Computer Interfaces (cBCI) open new opportunities for such seamless and fine-grained control, beyond medical or lab experimentations. Although usual BCIs rely on medical grade devices, we have previously shown the feasibility of cBCI interface with cheap consumer-grade EEG devices, relying on \textcolor{red}{\bf [complete sentence here]} \cite{}. We shall expand this method, using a continuum of entropy-based attention metrics, to compute in real-time the level of attention around each word presented and control the pace of words accordingly, in a continuous optimization process.

This article is organized as follows. In Section \ref{related_work}, we review the existing literature first on Rapid Serial Visual Presentation (RSVP), and then on Brain-Computer Interfaces (BCI). In Section \ref{method}, we introduce our research hypotheses. In Section \ref{apparatus}, we set technical assumptions and present the {\it brain speed reader} apparatus. In Section \ref{exp_protocol}, we detail the experimental protocol. In Section \ref{data}, we detailed the acquired data and post-experiment treatment. In Section \ref{results}, we present our results, which we then discuss in Section \ref{discussion}, before concluding in Section \ref{conclusion}.


